% ghcmodHappyHaskellProgram-Dg.tex
\begin{hcarentry}[updated]{ghc-mod --- Happy Haskell Programming}
\report{Daniel Gr\"ober}%05/15
\status{open source, actively developed}
\makeheader

\texttt{ghc-mod} is both a backend program for enhancing editors and other kinds
of development environments with support for Haskell, and an Emacs package
providing the user facing functionality, internally called \texttt{ghc} for
historical reasons. Other people have also developed numerous front ends for Vim
and there also exist some for Atom and a few other proprietary editors.

After a period of declining activity, development has been picking up pace again
since Daniel Gr\"ober took over as maintainer. Most changes during versions
5.0.0--5.2.1.2 consisted only of fixes and internal cleanup work, but for the
past four months, vastly improved Cabal support has been in the works and is now
starting to stabilize.

This work is a major step forward in terms of how well ghc-mod's suggestions
reflect what \texttt{cabal build} would report, and should also allow ghc-mod's
other features to work even in more complicated Cabal setups.

Daniel Gr\"ober has been accepted for a summer internship at IIJ Innovation
Institute's Research Laboratory working on \texttt{ghc-mod} for two months
(August--September). He will be working on:
\begin{compactitem}

  \item adding GHCi-like interactive code execution, to bring \texttt{ghc-mod} up
    to feature parity with GHCi and beyond,

  \item investigating how to best cooperate with \texttt{ide-backend},

  \item adding a network interface to make using ghc-mod in other projects
    easier, and

  \item if time allows, cleaning up the Emacs frontend to be more user-friendly
    and in line with Emacs' conventions.
\end{compactitem}

The goal of this work is to make \texttt{ghc-mod} the obvious choice for anyone
implementing Haskell support for a development environment and improving
\texttt{ghc-mod}'s overall feature set and reliability in order to give new as
well as experienced Haskell developers the best possible experience.

Right now \texttt{ghc-mod} has only one core developer and only a handful of
occasional drive-by contributors. If \textit{you} want to help make Haskell
development even more fun come and join us!

\FurtherReading
  \url{https://github.com/kazu-yamamoto/ghc-mod}
\end{hcarentry}
